\documentclass[11pt]{article}
\usepackage{cite}
%-------------preamble begins-------------------
\usepackage[textwidth=6.2in,textheight=9.5in]{geometry}
\usepackage{amsfonts}
\usepackage{lmodern}
\usepackage{amsmath}
\usepackage{fancyhdr} 
\usepackage{array,booktabs}
\usepackage{csvsimple}
\usepackage{pdfpages}
\usepackage{mathtools, nccmath}
\usepackage{refcheck}
\usepackage{titlesec}
\usepackage{appendix}
\setcounter{secnumdepth}{4}
\titleformat{\paragraph}
{\normalfont\normalsize\bfseries}{\theparagraph}{1em}{}
\titlespacing*{\paragraph}
{0pt}{3.25ex plus 1ex minus .2ex}{1.5ex plus .2ex}
\newcommand{\pl}{\partial}
\newcommand{\D}{\Delta}
\fancyfoot{\thepage}
\renewcommand*\familydefault{\sfdefault}
\sffamily
\newcommand{\rd}{\mathrm{d}}
\newcommand{\al}{\mathrm{\alpha}}
\title{Stability and Error Analysis for the Numerical Solution of the Non Linear Heat Flow Equation using Finite Differences\\ - draft v0.5}
\author{Andy Legg}
\date{May 2017}
%
%-------------preamble ends---------------------
\begin{document}
%-------------start of document body----------
\maketitle
\section{Summary}
This document provides an analysis of stability requirements and error estimates for the non linear heat equation using Finite Differences (FD).
\section{Introduction}
To determine stability and error estimates initially consider the problem in one dimension, in a fixed reference frame as follows: 
\begin{equation}
\label{eq:pde}
u_t = (\alpha(u)u_x)_x \text{ for } x \in (0,1), t\in (0,T]
\end{equation}
with initial condition $u(x,0) = g(x) \text { for } x\in [0,1] $ and boundary conditions $u(0,t) = u_l(t), u(1,t)= u_r(t) \text{ for } t \in(0,T]$ where $g(x) $ is a given function such that $g$ is smooth and decreasing for $\ \in (0,1)$. Further assume that $\alpha(u)$ is a smooth function, $\alpha(u) > 0 \;\forall\; x\in(0,1), \;t\in (0,T]$. In addition assume $\exists $ constants $\;\alpha_{0}, \alpha_{1} \; \in \mathbb{R}\; \ni  0 < \alpha_{0}\leq\alpha(u)\leq\alpha_{1}$. 
\subsection{FD Schemes}
We discretize equation (1) by dividing the interval $[0,1]$ into $J$ subintervals with $$0 = x_1<x_2<.......<x_J<x_{J+1}=1$$ 
Define $r = \Delta{t}/\Delta{x}^2$. In the notation used the superscript $n$ represents the timestep, $U_j^n$ represents the approximate numerical solution and $u_j^n = u(j\Delta{x}, n\Delta{t})$ the exact solution to the PDE (1) at the point $(j\Delta{x}, n\Delta{t})$. Two difference schemes will be considered, explicit and implicit. In both cases a half centred scheme is used for the $x$ coordinate and forward and backward time differences respectively for the explicit and implicit difference equations. These schemes are set out below.
\begin{equation}
\label{eq:expdiff}
U_j^{n+1} = U_j^n + r(\alpha_{j+1/2}^nU_{j+1}^n - (\alpha_{j+1/2}^n+\alpha_{j-1/2}^n)U_j^n+\alpha_{j-1/2}^nU_{j-1}^n)
\end{equation}
\begin{equation}
\label{eq:impdiff}
U_j^{n+1} = U_j^n + r(\alpha_{j+1/2}^nU_{j+1}^{n+1} - (\alpha_{j+1/2}^n+\alpha_{j-1/2}^n)U_j^{n+1}+\alpha_{j-1/2}^nU_{j-1}^{n+1})
\end{equation}
Where  $ \alpha^n_{j+1/2} = ( U^n_{j+1}+U^n_j)/2, \alpha^n_{j-1/2} = ( U^n_{j}+U^n_{j-1})/2$ for $j = 1,...,J+1$ and $ n \in \mathbb{N}$ with $n > 0$.
%
%
\subsection{Maximum Principle}
Define region $R = \{(x,t):x\in [0,1], t\in(0,T]\}$ with $\pl{R}$ as the boundary of $R$. Suppose that $u\in{C(R)}$ and is a solution of $(\ref{eq:pde})$
with $u_t,\,u_x,\,u_{xx}\, \in\,C\left((0,1)\times(0,T]\right)$. Then $u$ satisfies
\begin{equation}
\label{eq:maxpri}
\inf\limits_{(x,t) \in\pl{R}}\left(g(x),u_l(t),u_r(t)\right)   \leq u(x,t)\leq \sup\limits_{(x,t)\in\pl{R}}\left(g(x),u_l(t),u_r(t)\right)
\end{equation}
This means that for $(\ref{eq:pde})$ the maximum and minimum values of $u$ will always occur on the boundary. A similar result for the discrete form in the difference schemes $(\ref{eq:expdiff})$ and $(\ref{eq:impdiff})$ may be stated
\begin{equation}
\label{eq:dismaxpri}
\min\limits_{(x_j,t_n) \in\pl{R_{\D }}}\left(g(x_j),u_l(x_1,t_n),u_r(x_{J+1},t_n)\right)   \leq u(x_j,t_n)\leq \max\limits_{(x_j,t_n)\in\pl{R_{\D}}}\left(g(x_j),u_l(x_1,t_n),u_r(x_{J+1},t_n)\right)
\end{equation}
The proof of these results is not given here but may be found in \cite[p. 188-192]{Tveito1}
%
%
\subsection{Truncation Error}
\subsubsection{General case}
It is shown in Appendix A that the Truncation error for the explicit scheme defined in equation $(\ref{eq:expdiff})$, the truncation error may be expressed as
\begin{equation}
\label{eq:exptru}
\begin{split}
&\tau(x_j,t_n) = (1/2)\D{t}A + (1/2)\D{x^2}B + O(\D{t^2}) + O(\D{x^4})\\ 
&\begin{aligned}
\text{where\qquad}
A &= u_{tt}(x_j,t_n)\\
B &= \frac{ 1 }{6 }\al(u(x_j,t_n))u_{xxxx}(x_j,t_n)+ \frac{1}{3} \al_x(u(x_j,t_n ))u_{xxx}(x_j,t_n ) \\ &+\frac{1}{2}\al_{xx}(u(x_j,t_n))u_{xx}(x_j,t_n) +\frac{1}{3}\al_{xxx}(u(x_j,t_n))u_{x}(x_j,t_n)
\end{aligned}
\end{split}
\end{equation}
In addition following the same process outlined in Appendix A it can be concluded that for the implicit scheme $(\ref{eq:impdiff})$ the truncation error is
\begin{equation}
\label{eq:imptru}
    \begin{split}
        &\tau(x_j,t_{n+1}) = (1/2)\D{t}C + (1/2)\D{x^2}D  +  O(\D{t^2}) + O(\D{x^4})\\ 
        &\begin{aligned}
            \text{where\qquad}
            C &=  u_{tt}(x_j,t_{n+1})\\
            D &= \frac{ 1 }{6 }\al(u(x_j,t_n))u_{xxxx}(x_j,t_{n+1})+ \frac{1}{3} \al_x(u(x_j,t_{n} ))u_{xxx}(x_j,t_{n+1} ) \\ &+\frac{1}{2}\al_{xx}(u(x_j,t_{n})u_{xx}(x_j,t_{n+1})) +\frac{1}{3}\al_{xxx}(u(x_j,t_n))u_{x}(x_j,t_{n+1})
        \end{aligned}
    \end{split}
\end{equation}
%
%
\subsubsection{Examples}
Linear Heat Flow Equation\\
Consider case $\al(u)=1$ equation $\ref{eq:pde}$ becomes $u_t=u_{xx}$. Then the derivatives of $\al(u)$ wrt $x$ are zero. The truncation error equation $(\ref{eq:exptru})$ becomes
\begin{equation}
\label{eq:hetru}
\tau_j^n = (1/2)\D{t}u_{tt}(x_j,t_n) + (1/12) \D{x^2}u_{xxxx}(x_j,t^n) +O(\D{t^2})+ O(\D{x^4}) 
\end{equation}
Since $$u_t = u_{xx}, \text{ then }u_{tt} = u_{xxt} = u_{txx} = u_{xxxx}$$ 
we can write $(\ref{eq:hetru})$ as
\begin{equation}
\label{eq:hetru1}
\tau_j^n =\left ((1/2)\D{t} + (1/12) \D{x^2}\right)u_{xxxx}(x_j,t_n) +O(\D{t^2})+ O(\D{x^4}) 
\end{equation}
Porous Medium Equation\\
Consider case $\al(u)=u$ equation $\ref{eq:pde}$ is then $u_t=({uu_x})_x$. The truncation error equation $\ref{eq:exptru}$ becomes
\begin{equation}
\label{eq:pmetru}
\begin{split}
&\tau_j^n = (1/2)\D{t}u_{tt}(x_j,t_n) + \\
&\begin{aligned}
&(1/2) \D{x^2}\left(  (1/6)u(x_j,t_n)u_{xxxx}(x_j,t_n) + (2/3)u_x(x_j,t_n)u_{xxx}(x_j,t_n) + (1/2) (u_{xx}(x_j,t_n))^2  \right) \\
+&O(\D{t^2})+ O(\D{x^4})
\end{aligned}
\end{split}
\end{equation}
To make use of this result in the error analysis it will be necessary to express the time derivative term as spatial derivatives. 
Since $$u_t= (uu_x)_x$$ it can be shown that
$$u_{tt} = \left( (uu_x)_xu_x + u(uu_x)_{xx}  \right)_x  = \left(u(uu_x)_x     \right)_{xx}$$
%
%
%
\subsection{Stability Analysis}
As a first step consider equation \ref{eq:pde} with a forcing term $f(x,t)$, where $f(x,t)$ is a smooth continuous function for $x \in (0,1), t\in (0,T]$
\begin{equation}
\label{eq:pdefor}
u_t = (\al(u)u_x)_x+f(x,t)
\end{equation}
To discretize this equation the explicit difference scheme \ref{eq:expdiff} is modified as follows:
\begin{equation*}
U_j^{n+1} = U_j^n + r(\alpha_{j+1/2}^nU_{j+1}^n - (\alpha_{j+1/2}^n+\alpha_{j-1/2}^n)U_j^n+\alpha_{j-1/2}^nU_{j-1}^n) + \D{t}f_j^n
\end{equation*}
Rewriting this equation
\begin{equation}
\label{eq:expfor}
U_j^{n+1} = U_j^n(1 - r(\alpha_{j+1/2}^n+\alpha_{j-1/2}^n)) + r\alpha_{j+1/2}^nU_{j+1}^n +r\alpha_{j-1/2}^nU_{j-1}^n) + \D{t}f_j^n 
\end{equation}
 Noting that $r > 0$ and that ${\al}> 0 $ 
 since $U_j^n > 0\;\forall \;j, n $ and defining  $\underline{U}^n={(U_1^n,U_2^n,...,U_{J+1}^n)}^T$, $\|{\underline{U}^n}\|_{\D,\infty} = \max\limits_{1\leq{j}\leq{J+1}   }{U_j^n} = M^n $ and
$ \|{\underline{f^n}\|_{\D,\infty}} = \max\limits_{1\leq{j}\leq{J+1}   }{f_j^n} = F^n $.\\
If we have $1 - r(\alpha_{j+1/2}^n+\alpha_{j-1/2}^n) \geq 0$ then equation \ref{eq:expfor} gives
\begin{equation*}
U_j^{n+1} \leq M^n(1 - r(\alpha_{j+1/2}^n+\alpha_{j-1/2}^n)) + r\alpha_{j+1/2}^nM^n +r\alpha_{j-1/2}^nM^n + \D{t}F^n 
\end{equation*}
simplifying this gives 
\begin{equation}
\label{eq:maxpri}
U_j^{n+1} \leq M^n + \D{t}F^n 
\end{equation}
Taking the maximum wrt $j$ on the LHS of the inequality (\ref{eq:maxpri})
\begin{equation*}
M^{n+1} \leq M^n + \D{t}F^n
\end{equation*}
Hence
\begin{equation*}
M^n \leq M^0 + \D{t}\sum_{k=1}^n{F^{k-1}}
\end{equation*}
which can be written
\begin{equation}
\label{eq:expsta}
M^n \leq M^0 + n\D{t}\max_{1\leq{k}\leq{n}}{F^{k-1}}
\end{equation}
where $M^0$ can be determined from the initial data, recall $u(x,0) = g(x)$, so\\ $M^0 = \|\underline{g}\|_{\D,\infty} = \max\limits_{1\leq{j}\leq{J+1}}g_j$.\\
In obtaining this result it was necessary that  
\begin{equation}
\label{eq:expsta1}
(1 - r(\alpha_{j+1/2}^n+\alpha_{j-1/2}^n)) \geq 0  
\end{equation}
This condition is required for the stability of any approximate solution to equation $(\ref{eq:expdiff})$.\\ If $f = 0$ inequality $(\ref{eq:maxpri})$ gives 
\begin{equation*}
U_j^n \leq M^0
\end{equation*}
ensuring the numerical solution is bounded by the maximum value of $u \;\forall \; t$
Re-writing inequality $(\ref{eq:expsta})$
\begin{equation*}
r(\alpha_{j+1/2}^n+\alpha_{j-1/2}^n) \leq 1
\end{equation*}
The maximum value of the LHS of this inequality will be at $n=0$ so $ 2M^0r \leq 1$ which since $r = \D{t}/\D{x^2}$ gives the following result for the stability of the explicit scheme:
\begin{equation}
\label{eq:expsta1}
\D{t} \leq \frac{\D{x^2}}{2M^0}
\end{equation}
To address the stability requirements for the implicit scheme discretize equation $(\ref{eq:pdefor})$ using a modified equation $(\ref{eq:impdiff})$ as follows:
\begin{equation*}
U_j^{n+1} = U_j^n + r(\alpha_{j+1/2}^nU_{j+1}^{n+1} - (\alpha_{j+1/2}^n+\alpha_{j-1/2}^n)U_j^{n+1}+\alpha_{j-1/2}^nU_{j-1}^{n+1}) + \D{t}f_j^{n+1}
\end{equation*}
Rearranging
\begin{equation*}
U_j^{n+1}( 1+  (\alpha_{j+1/2}^n+\alpha_{j-1/2}^n) ) = U_j^n + r(\alpha_{j+1/2}^nU_{j+1}^{n+1} +\alpha_{j-1/2}^nU_{j-1}^{n+1}) + \D{t}f_j^{n+1}
\end{equation*}
therefore
\begin{equation}
\label{eq:impmax}
U_j^{n+1}( 1+  (\alpha_{j+1/2}^n+\alpha_{j-1/2}^n) ) \leq M^n + 2rM^nM^{n+1} + \D{t}F^{n+1}
\end{equation}
For the maximum value of the LHS of inequality $(\ref{eq:impmax})$
\begin{equation*} 
M^{n+1}(1+2rM^n) \leq M^n + 2rM^nM^{n+1} + \D{t}F^{n+1}
\end{equation*}
so that
\begin{equation*} 
M^{n+1} \leq M^n + \D{t}F^{n+1}
\end{equation*}
which implies
\begin{equation*}
M^n \leq M^0 + \D{t}\sum_{k=1}^n{F^k}
\end{equation*}
which can be written
\begin{equation}
\label{eq:impsta}
M^n \leq M^0 + n\D{t}\max_{1\leq{k}\leq{n}}{F^k}
\end{equation}
Note the similarity to inequality $(\ref{eq:expsta})$ but in this case there are no conditions on $r$ hence the implicit scheme is unconditionally stable.
%
%
%
\subsection{Error Analysis}
Consider the explicit scheme given in equation $(\ref{eq:expdiff})$
\begin{equation*}
U_j^{n+1} = U_j^n + r(\alpha_{j+1/2}^nU_{j+1}^n - (\alpha_{j+1/2}^n+\alpha_{j-1/2}^n)U_j^n+\alpha_{j-1/2}^nU_{j-1}^n)
\end{equation*}
Substitute exact solution $u_j^n$ into this equation gives 
\begin{equation}
\label{eq:truerr}
\begin{split}
u_j^{n+1} = u_j^n + r( \bar{\alpha}_{j+1/2}^nu_{j+1}^n -
 (\bar{\alpha}_{j+1/2}^n+\bar{\alpha}_{j-1/2}^n)u_j^n+
 \bar{\alpha}_{j-1/2}^nu_{j-1}^n) + \D{t}\tau_j^n
\end{split}
\end{equation}
where $\tau_j^n$ is the truncation error and  $ \bar{\alpha}^n_{j+\frac{1}{2}} = \frac{1}{2}( u^n_{j+1}+u^n_j), \bar{\alpha}^n_{j-\frac{1}{2}} = \frac{1}{2}( u^n_{j}+u^n_{j-1})$.
%From the difference equation (2) we have
%\begin{equation*}
%U_j^{n+1} = U_j^n + r(\alpha_{j+1/2}^nU_{j+1}^{n+1} - (\alpha_{j+1/2}^n+\alpha_{j-1/2}^n)U_j^{n+1}+\alpha_{j-1/2}^nU_{j-1}^{n+1})
%\end{equation*}
%Where $U_j^n$ is the approximate solution, $\al_{j+1/2}^n\;,\al_{j-1/2}^n $ as previously defined.
Define error in solution at point $(j\D{x},n\D{t})$ as $e_j^n = u_j^n-U_j^n$ and subtracting equation $(\ref{eq:expdiff})$ from $(\ref{eq:truerr})$ gives
\begin{equation}
\label{eq:experr1}
\begin{split}
e_j^{n+1} = e_j^n + \D{t}\tau_j^n +   r( \bar{\alpha}_{j+1/2}^nu_{j+1}^n - (\bar{\alpha}_{j+1/2}^n+\bar{\alpha}_{j-1/2}^n)u_j^n+\bar{\alpha}_{j-1/2}^nu_{j-1}^n) - \\ r(\alpha_{j+1/2}^nU_{j+1}^n - (\alpha_{j+1/2}^n+\alpha_{j-1/2}^n)U_j^n+\alpha_{j-1/2}^nU_{j-1}^n) 
\end{split}
\end{equation}
In order to simplify equation $( \ref{eq:experr1})$ express $ \bar{\al}_{j+/-1/2}^n$ in terms of $\al_{j+/-1/2}^n$. 
To do this let $u_{j+/-1/2}^n = U_{j+/-1/2}^n +\D{U_{j+1/2}^n}$ 
where $\D{U_{j+1/2}^n} = (1/2)(U_{j+1}^n-U_j^n)$ and expand the bar terms as follows:\\
\begin{equation*}
\bar{\al}_{J+1/2}^n = \al(u_{J+1/2}^n) = \al(U_{j+1/2}^n + \D{U_{j+1/2}}) = \al(U_{j+1/2}^n)  + \D{U_{j+1/2}^n}\frac{ \pl{\al_{ j+1/2}^n } } { \pl{U} } + 
  \frac{(\D{U_{j+1/2}^n})^2}{2}\frac{ \pl^2{\al(\xi }_1 )} { \pl{U}^2 }
\end{equation*}
where $\xi_1\; \in\; (U_j,U_{j+1}) $ therefore
\begin{equation*}
\bar{\al}_{J+1/2}^n =   \al_{j+1/2}^n+ (1/2)(U_{j+1}^n-U_j^n)\frac{\al_{j+1}^n-\al_j^n } {U_{j+1}^n-U_j^n } + 
  \frac{(\D{U_{j+1/2}^n})^2}{2}\frac{ \pl^2{\al(\xi }_1 )} { \pl{U}^2 } 
\end{equation*}
giving
\begin{equation}
\label{eq:alexp1}
\bar{\al}_{j+1/2}^n = \al_{j+1/2}^n+ (1/2)(\al_{j+1}^n-\al_j^n ) +   \frac{(\D{U_{j+1/2}^n})^2}{2}\frac{ \pl^2{\al(\xi }_1 )} { \pl{U}^2 }
\end{equation}
and so
\begin{equation}
\label{eq:alexp2}
\bar{\al}_{j-1/2}^n   = \al_{j-1/2}^n+ (1/2)(\al_{j}^n-\al_{j-1}^n )  +  \frac{(\D{U_{j-1/2}^n})^2}{2}\frac{ \pl^2{\al(\xi }_2 )} { \pl{U}^2 }
\end{equation}
Where $\xi_2\; \in\; (U_{j-1},U_j) $ and noting that $\tau_j^n \sim e_j^n \sim \D{u_j^n} =O( \D{x^2}) $ implies $\sim \D{u}^2 \sim O(\D{x}^4)$.
Substitute $(\ref{eq:alexp1})$ and $(\ref{eq:alexp2})$ into the equation $(\ref{eq:experr1})$ gives
\begin{equation}
\label{eq:experr2}
e_j^{n+1} = e_j^n + \D{t}\tau_j^n +  r(\alpha_{j+1/2}^ne_{j+1}^{n+1} - (\alpha_{j+1/2}^n+\alpha_{j-1/2}^n)e_j^{n+1}+\alpha_{j-1/2}^ne_{j-1}^{n+1}) + R_j^n
\end{equation}
where $R_j^n$ is a residual term $\ni$ 
\begin{equation} 
\label{eq:residual1}
R_j^n = (1/2)r\left(  (\al_{j+1}^n-\al_j^n)u_{j+1}^n-( \al_{j+1}^n-\al_{j-1}^n)u_j^n  + (\al_j^n-\al_{j-1}^n)u_{j-1}^n     + O(\D{x^4}) \right)  
\end{equation}
Expanding the terms in $(\ref{eq:residual1}), \;\al_{j+/-1}^n$, around $\al_j^n$ and $u_{j+/-1}^n$ around $x_j^n$ and simplifying it can be shown that
\begin{equation} 
R_j^n = (1/2)r\left(\D{x^3}\left(   \frac{ \pl{\al_j^n} }{ \pl{x }} \frac{ \pl^2{u_j^n} }{ \pl{x^2 }} +  \frac{  \pl^2{\al_j^n}  }{ \pl{x^2 } }   \frac{ \pl{u_j^n} }{ \pl{x}}  \right )    + O(\D{x^4}  )\right) 
\end{equation}  
since $r=\D{t}/\D(x^2)$ then
\begin{equation} 
\label{eq:residual2}
 R_j^n = (1/2)\D{t}\left(\D{x}\left(   \frac{ \pl{\al_j^n} }{ \pl{x }} \frac{ \pl^2{u_j^n} }{ \pl{x^2 }} +  \frac{  \pl^2{\al_j^n}  }{ \pl{x^2 } }   \frac{ \pl{u_j^n} }{ \pl{x}}  \right )    + O(\D{x^2}  )\right) 
\end{equation} 
This may be written as
\begin{equation} 
\label{eq:residual3}
 R(x_j,t_n) = (1/2)\D{t}\left(\D{x}\left(  \al_x(u(x_j,t_n))u_{xx}(x_j,t_n) +  \al_{xx}(u(x_j,t_n))u_{x}(x_j,t_n)              \right )    + O(\D{x^2}  )\right) 
\end{equation} 
A consequence of this result in equation $(\ref{eq:residual2} )$ is that whilst the difference scheme for the linear linear heat flow equation $(\ref{eq:expdiff})$ derived from $(\ref{eq:pde})$ with $\al(u) = 1$ is second order in space, the non linear case will in general be first order.\\
Applying the result proved in the stability section, namely for a difference scheme of type shown in $(\ref{eq:expfor})$ 
\begin{equation*}
U_j^{n+1} = U_j^n(1 - r(\alpha_{j+1/2}^n+\alpha_{j-1/2}^n)) + r\alpha_{j+1/2}^nU_{j+1}^n +r\alpha_{j-1/2}^nU_{j-1}^n) + \D{t}f_j^n 
\end{equation*}
then the result bounding $U_j^n$ from equation $(\ref{eq:expsta})$ will apply viz.
\begin{equation*}
M^n \leq M^0 + n\D{t}\max_{1\leq{k}\leq{n}}{F^{k-1}}
\end{equation*}
Where $M^n = \|{\underline{U}^n}\|_{\D,\infty} = \max\limits_{1\leq{j}\leq{J+1}   }{U_j^n}  $ and $ F^n = \|{\underline{f^n}\|_{\D,\infty}} = \max\limits_{1\leq{j}\leq{J+1}   }{f_j^n} $.\\
Compare equation $(\ref{eq:experr2})$ with equation $(\ref{eq:expfor})$ and make the error terms $\tau_j^n +R_j^n$ in the former equation equate to the forcing term $f_j^n$ in the latter, so if $R_j^n = \D{t}S_j^n$ then $f_j^n = \tau_j^n + S_j^n$. 
Define $\underline{e}^n={(e_1^n,e_2^n,...,e_{J+1}^n)}^T$ so $\|{\underline{e}^n}\|_{\D,\infty} = \max\limits_{1\leq{j}\leq{J+1}   }{e_j^n} = E^n $\\
If $e$ is substituted for $U$ in $(\ref{eq:experr2})$ then using $(\ref{eq:expsta})$
\begin{equation*}
E^n \leq E^0 + n\D{t}\max_{1\leq{k}\leq{n}}F^{k-1}
\end{equation*}  
We know $E^0 = 0$ as $u_j^0 - U_j^0 = 0\; \forall\; {j}$ \;from the initial conditions.
Hence we derive the error bound for the explicit difference scheme $(\ref{eq:expdiff})$
\begin{equation}
\label{eq:experr3}
 E^n \leq  n\D{t}\max_{1\leq{k}\leq{n}}F^{k-1}
\end{equation}
In order to quantify the error it is necessary to determine an upper bound for $f(x_j,t_n) = \tau(x_j,t_n) + S(x_j,t_n)$. Consider 
\begin{equation} 
\label{eq:residual4}
 S(x_j,t_n) = (1/2)\left(\D{x}\left(  \al_x(u(x_j,t_n))u_{xx}(x_j,t_n) +  \al_{xx}(u(x_j,t_n))u_{x}(x_j,t_n)              \right )    + O(\D{x^2}  )\right) 
\end{equation} 
and from equation $(\ref{eq:exptru})$
\begin{equation*}
\begin{split}
&\tau(x_j,t_n) = (1/2)\D{t}A + (1/2)\D{x^2}B + O(\D{t^2}) + O(\D{x^4})\\ 
&\begin{aligned}
\text{where\qquad}
A &= u_{tt}(x_j,t_n)\\
B &= \frac{ 1 }{6 }\al(u(x_j,t_n))u_{xxxx}(x_j,t_n)+ \frac{1}{3} \al_x(u(x_j,t_n ))u_{xxx}(x_j,t_n ) \\ &+\frac{1}{2}\al_{xx}(u(x_j,t_n))u_{xx}(x_j,t_n) +\frac{1}{3}\al_{xxx}(u(x_j,t_n))u_{x}(x_j,t_n)
\end{aligned}
\end{split}
\end{equation*}
Noting that the above terms for $\tau$ and $S$ are formed from the derivates of $u$ it will be necessary to determine bounds on the derivatives to determine the error.
Consider equation $(\ref{eq:expdiff})$ and with $M^n = \|{\underline{U}^n}\|_{\D,\infty} = \max\limits_{1\leq{j}\leq{J+1}   }{U_j^n}  $.
Then 
$$U_j^{n+1} \leq M^n(1 -r  (\alpha_{j+1/2}^n+\alpha_{j-1/2}^n)) + r\alpha_{j+1/2}^nM^n +r\alpha_{j-1/2}^nM^n)$$
therefore $\forall\;j= 1, 2....,J+1 \text{ and }n = 0, 1, 2,....,N$
\begin{equation}
\label{eq:maxpri2}
U_j^{n+1} \leq M^n
\end{equation}
This implies
\begin{equation}
\label{eq:maxpri20}
M^{n+1} \leq  M^n
\end{equation}
so
$$ M^1 \leq M^0, M^2 \leq M^1 \leq M^0$$
and so on for increasing n giving
\begin{equation}
\label{eq:maxpri21}
M^n \leq M^0
\end{equation}
This result is consistent with inequality $(\ref{eq:dismaxpri})$ 
$$ M^0 = \max\limits_{1\leq{j}\leq{J+1}}u(x_j,0)  = \|\underline{g}\|_{\D,\infty}$$
where $\underline{g} = (g(x_1), g(x_2),.....,g(x_{J+1}))^T$ contains the initial values of $u$
It has been possible to relate the maximum values of $u$ to the initial data. The relationship between the maximum value of the derivatives of u and the initial data will depend upon each example. \\
$\mathbf{Note\; to\; Tristan}$\\
In order to use the error bound using equations $(\ref{eq:exptru})$ and  $(\ref{eq:residual4})$ in the examples I wanted to show the following:
$$\|\frac{\pl }{\pl{x}}\underline{U}^{n+1}\|_{\D,\infty} \leq \|\frac{\pl }{\pl{x}}\underline{U}^n \|_{\D,\infty}$$
This would imply $$\|\frac{\pl }{\pl{x} }\underline{U}^{n}\|_{\D,\infty} \leq \|\frac{\pl }{\pl{x} }\underline{g}\|_{\D,\infty}$$
as well as for the higher derivatives.\\
Questions\\
\begin{enumerate}
\item{Do you think this is feasible/does the approach make sense?}
\item{If yes then do you have any ideas on the approach that should be used to prove this?}
\item{Is there a better way to calculate the error bound}
\end{enumerate}

%
%THE FOLLOWING SECTION TO 2.6 EXAMPLES - INCORRECT/INCOMPLETE
%WIP
%
%Using the max principle $(\ref{eq:dismaxpri})$  there are no local max/min in $R$. The solution $u$ preserves the same characteristics as the initial data. Since $g$ is decreasing
%\begin{equation}
%\label{eq:maxpri5}
%U_j^n \leq U_{j-1}^n
%\end{equation}
%and
%\begin{equation}
%\label{eq:maxpri6}
%U_j^{n+1} \leq U_{j-1}^{n+1}
%\end{equation}
%Adding inequalities $\ref{eq:maxpri5}$ and  $\ref{eq:maxpri6}$
%$$U_j^n + U_j^{n+1} \leq U_{j-1}^n + U_{j-1}^{n+1}$$
%therefore
%$$U_j^n - U_{j-1}^{n} \leq U_{j-1}^{n+1} - U_j^{n+1}$$
%
%therefore for $\D{x} > 0$
%$$\frac{U_j^{n+1} -U_{j-1}^{n+1}}{\D{x}}  \leq \frac{U_j^n-U_{j-1}^n}{\D{x}}  $$
%Taking the limit as $\D{x}$ and $\D{t} \to 0$
%$$\lim_{   \D{x},\D{t}\to 0       }\left( \frac{U_j^{n+1} -U_{j-1}^{n+1}}{\D{x}} \right)  \leq \lim_{ \D{x},\D{t}\to 0   }\left( \frac{U_j^n-U_{j-1}^n}{\D{x}} \right)$$
%Therefore assuming $U_j^n \to u(x_j,t_n)$
%\begin{equation}
%\label{eq:maxprider1}
%u_x(x_j,t_{n+1}) \leq u_x(x_j,t_n)
%\end{equation}
%So in particular from inequality $(\ref{eq:maxprider1})$
%$$ u_x(x_j,t_{n+1}) \leq u_x(x_j,t_0) $$
%which implies
%\begin{equation}
%\label{eq:maxprider2}
%\|\underline {u_x}^n\|_{\D,\infty}  \leq \|\underline{u_x}^0\|_{\D,\infty} = \|\underline{g_x}\|_{\D,\infty}
%\end{equation}
%The same argument can be applied to higher derivatives giving 
%\begin{equation}
%\label{eq:maxprider3}
%\|\underline {u_{px}}^n\|_{\D,\infty}  \leq \|\underline{u_{px}}^0\|_{\D,\infty}= \|\underline{g_{px}}\|_{\D,\infty}
%\end{equation}
%Where $ p = 1,2,3...$ and $u_{px}$ means $u_x,u_{xx}, u_{xxx},....$
%Therefore for our PDE $(\ref{eq:pde})$ applying the results from $(\ref{eq:maxpri21})$ and $(\ref{eq:maxprider3}),\; \underline{u}^n$ and any associated derivatives are bounded by the initial data.


\subsection{Examples}
\setcounter{tocdepth}{5}
%\titleformat{\paragraph}[runin]{\normalfont\normalsize}{\theparagraph}{1em}{}[. \mbox{}]
\subsubsection{Heat Equation with first order diffusion coefficient - Porous Medium Equation }
%PME equation\\
Take $\al(u) = u $ gives equation $(\ref{eq:pde})$ as
\begin{equation} 
u_t = (uu_x)_x
\end{equation}
The exact solution of this equation is well known \cite{Baines2016} and in 1D is
\begin{equation}
\label{eq:PMEsol}
u(x,t) = \frac{1}{6t^{ \frac{1}{3} }  } \left(  1-\left(    \frac{x }{t^{\frac{1}{3}  }  } \right)^2   \right)_{+}
\end{equation}
$$\text{where the notation } (s)_{+ }= max\{s,0\} $$
we define the initial condition from $(\ref{eq:PMEsol})$ with $t_0 = 1$ giving
\begin{equation}
\label{eq:PMEini}
 g(x) = (1/6)(1-x^2)
\end{equation}
and the boundary conditions from $(\ref{eq:PMEsol})$ for $t > 1$ as follows:
$$ u(0,t) = 1/(6t^{1/3}) \text{ and } u(1,t) = 1/(6t^{1/3})\left(1-1/(t^{2/3})\right)$$
For the explicit difference scheme $(\ref{eq:expdiff})$ for stability use inequality $(\ref{eq:expsta1})$
Suppose the domain is divided into $J$ equal intervals then since $dx = 1/J $, use equation $(\ref{eq:PMEini})$ to determine $M^0$
$$M^0 = \|g\|_{\D,\infty} = 1/6$$
So for stability $$\D{t} \leq \D{x^2}/(2M^0) = 3/J^2$$
The following table compares the numerical solution $U_j^N$ after a computation period of time $= 3 = Ndt$ with the exact solution $u(x_j,t_N)$ \\
\\
\csvautotabular{FD_PME_Resultsv_1.csv}\\
\\
\\Graphical output for case $J=40$ is given in figure $\ref{fig:instability1}$, it can be seen from the results that the solution is both stable and accurate with order $2$ convergence. This is better than expected from the theoretical results, the residual term from equation $(\ref{eq:residual2})$ predicted only first order convergence would be achieved.
\\The stability condition $(\ref{eq:expsta1})$ gives $dt<1.90E-03$ for $J=40$, when a value of $dt\geq2.30E-03$ is applied the explicit scheme becomes unstable as shown in figure $\ref{fig:instability}$. This shows that the derived stability condition is reasonably tight.\\
The table below shows comparative results for the implicit difference scheme $(\ref{eq:impdiff})$ after a computation period of time $= 3 = Ndt$ with the exact solution $u(x_j,t_N)$ \\
\\
\csvautotabular{implicit_1Dm_1.csv}\\
\\
Note the method produces second order convergence with $dt = 0.5dx^2$. Also that with a relatively large $dt$ for example $dt = 0.1$ the method remains stable. In this example there are only $N=20$ time steps and the method converges although at a much slower rate.\\
\\Graphical output for case $J=40$ is given in figure $\ref{fig:impdiff1}$

\subsubsection{Heat Equation with second order diffusion coefficient}
Take $\al(u) = u^2 $ gives equation $(\ref{eq:pde})$ as
\begin{equation} 
u_t = (u^2u_x)_x
\end{equation}
The exact solution of this equation is also well known \cite{vazquez2007} and in 1D is
\begin{equation}
\label{eq:PMEsol}
u(x,t) = \frac{1}{2t^{ \frac{1}{4} }  }\left(  1-\left(    \frac{x }{t^{\frac{1}{4}  }  } \right)^2  \right) ^{1/2}
\end{equation}
we define the initial condition from $(\ref{eq:PMEsol})$ with $t_0 = 1$ giving
\begin{equation}
\label{eq:PMEini}
 g(x) = (1/6)(1-x^2)^{1/2}
\end{equation}
and the boundary conditions from $(\ref{eq:PMEsol})$ for $t > 1$ as follows:
$$ u(0,t) = 1/(2t^{1/4}) \text{ and } u(1,t) = 1/(2t^{1/4})\left(1-1/(t^{1/2})\right)^{1/2}$$
For the explicit difference scheme and following the same procedure as in the previous example 
$$M^0 = \|g\|_{\D,\infty} = 1/4$$
So for stability $$\D{t} \leq \D{x^2}/(2M^0) = 2/J^2$$
The following table compares the numerical solution $U_j^N$ after a computation period of time $= 3 = Ndt$ with the exact solution $u(x_j,t_N)$ \\
\\
\\
\csvautotabular{FD_PME_Resultsv_2.csv}\\
\\
\\Graphical output for case $J=40$ is given in figure shown in figure $\ref{fig:instability2}$, it can be seen from the results that the solution is both stable and accurate with order $2$ convergence. Again this is better than expected from the theoretical results, the residual term from equation $(\ref{eq:residual2})$ predicted only first order convergence would be achieved.
\\The stability condition $(\ref{eq:expsta1})$ gives $dt<1.30E-03$ for $J=40$, when a value of $dt\geq1.41E-03$ is applied the explicit scheme becomes unstable as shown in figure $\ref{fig:instability3}$. This again shows that the derived stability condition is reasonably tight.



\newpage
\subsection{Figures}

%\begin{figure}
%\label{fig1}
% \centering 
% \includepdf{expdiffinst.pdf}
%\end{figure}
\begin{figure}[h!]
 \includepdf[scale=1.0]{expdiffinst1.pdf}
  \caption{Solution using explicit difference scheme $(\ref{eq:expdiff})$ with $\al(u) = u$ after N = $1600$ timesteps with $J = 40$ and $dt = 1.90E-03$. The initial position is shown in blue, the computed solution in red and the exact solution in green.}
  \label{fig:instability1}
\end{figure}
\newpage
\begin{figure}[h!]
 \includepdf[scale=1.0]{expdiffinst.pdf}
  \caption{Instability in explicit difference scheme $(\ref{eq:expdiff})$ with $\al(u) = u$ after N = $40$ timesteps with $J = 40$ and $dt = 2.30E-03$. The initial position is shown in blue, the computed solution in red and the exact solution in green.}
  \label{fig:instability}
\end{figure}
\newpage
\begin{figure}[h!]
 \includepdf[scale=1.0]{expdiffinst2.pdf}
  \caption{Solution using explicit difference scheme $(\ref{eq:expdiff})$ with $\al(u) = u^2$ after N = $800$ timesteps with $J = 40$ and $dt = 1.30E-03$. The initial position is shown in blue, the computed solution in red and the exact solution in green.}
  \label{fig:instability2}
\end{figure}
\newpage
\begin{figure}[h!]
 \includepdf[scale=1.0]{expdiffinst3.pdf}
  \caption{Instability in explicit difference scheme $(\ref{eq:expdiff})$ with $\al(u) = u^2$ after N = $70$ timesteps with $J = 40$ and $dt = 1.41E-03$. The initial position is shown in blue, the computed solution in red and the exact solution in green.}
  \label{fig:instability3}
\end{figure}
\newpage
\begin{figure}[h!]
 \includepdf[scale=1.0]{impdiff1.pdf}
  \caption{Solution using implicit difference scheme $(\ref{eq:impdiff})$ with $\al(u) = u$ after N = $3200$ timesteps with $J = 40$ and $dt = 3.13E-04$. The initial position is shown in blue, the computed solution in red and the exact solution in green.}
  \label{fig:impdiff1}
\end{figure}
\newpage
\nocite{Morton1}
\bibliography{RefLib}
\bibliographystyle{plain}
\newpage
\appendix
\begin{appendices}
\appendixpage
\section{Truncation Error}
Consider the explicit scheme defined in equation (2). Calculate the truncation error $\tau(j\Delta{x},n\Delta{t})=\tau_j^n$ at the point $(j\Delta{x},n\Delta{t})$ by substituting the exact solution $u(j\Delta{x},n\Delta{t}) = u_j^n$ into equation $(\ref{eq:expdiff})$ and defining $\bar\al_{j+1/2}^n = 1/2(\al(u(x_{j+1},t_n)) +\al(u(x_j,t_n))$ and \\$\bar\al_{j-1/2}^n = 1/2(\al(u(x_{j},t_n)) +\al(u(x_{j-1},t_n)))$
\begin{equation}
\label{eq:tru}
%\begin{split}
\tau_j^n=\frac{ u_j^{n+1} - u_j^n}{\Delta{t}} + \frac{(\bar\alpha_{j+1/2}^nu_{j+1}^n - (\bar\alpha_{j+1/2}^n+\bar\alpha_{j-1/2}^n)u_j^n+\bar\alpha_{j-1/2}^nu_{j-1}^n)} {\Delta{x^2}} 
%\end{split}
\end{equation}
First use Taylor's series to expand $u_j^{n+1}$ giving
\begin{equation*}
u_j^{n+1} -u_j^n = \D{t}\frac{\pl{u_j^n }}{\pl{t}} + \frac{\D{t^2}}{2}\frac{\pl^2{u_j^n }}{\pl{t^2}}+O(\D{t^3})
\end{equation*}
Next use Taylor's series to expand the remaining terms on the RHS of equation (4) around $x_j^n$.
\begin{equation*}
%\begin{split}
u_{j+1}^n = u_j^n + \D{x}\frac{\pl{u_j^n} } {\pl{x} } +\frac {\D{x^2}}{2}\frac{ \pl^2{u_j^n} } { \pl{x^2 }} + \frac {\D{x^3}}{6}\frac{ \pl^3{u_j^n} } { \pl{x^3 } } + \\ \frac {\D{x^4}}{24}\frac{ \pl^4{u_j^n} } { \pl{x^4 } } + O(\D{x^5})
%\end{split} 
\end{equation*}
\begin{equation*}
u_{j-1}^n = u_j^n - \D{x}\frac{\pl{u_j^n} } {\pl{x} } +\frac {\D{x^2}}{2}\frac{ \pl^2{u_j^n} } { \pl{x^2 }} - \frac {\D{x^3}}{6}\frac{ \pl^3{u_j^n} } { \pl{x^3 } } +\frac {\D{x^4}}{24}\frac{ \pl^4{u_j^n} } { \pl{x^4 } } + O(\D{x^5})
\end{equation*}
\begin{equation*}
\begin{split}
&\bar\alpha_{j+1/2}^n = (\bar\al_{j+1}^n+\bar\al_j^n)/2 = \\
&\begin{aligned}
&(1/2)\left(2\bar\al_j^n + \D{x}\frac{\pl{\bar\al_j^n} } {\pl{x} } + \frac {\D{x^2}}{2}\frac{ \pl^2{\bar\al_j^n} } { \pl{x^2 }} + \frac {\D{x^3}}{6}\frac{ \pl^3{\bar\al_j^n} } { \pl{x^3 } } + \frac {\D{x^4}}{24}\frac{ \pl^4{\bar\al_j^n} } { \pl{x^4 } } + O(\D{x^5}) \right) 
\end{aligned}
\end{split}
\end{equation*}
\begin{equation*}
\begin{split}
&\bar\alpha_{j-1/2}^n = (\bar\al_{j}^n+\bar\al_{j-1}^n)/2 = \\
&\begin{aligned}
&(1/2)\left(2\bar\al_j^n - \D{x}\frac{\pl{\bar\al_j^n} } {\pl{x} } +\frac {\D{x^2}}{2}\frac{ \pl^2{\bar\al_j^n} } { \pl{x^2 }} - \frac {\D{x^3}}{6}\frac{ \pl^3{\bar\al_j^n} } { \pl{x^3 } }+ \frac {\D{x^4}}{24}\frac{ \pl^4{\bar\al_j^n} } { \pl{x^4 } } + O(\D{x^5}) \right) 
\end{aligned}
\end{split}
\end{equation*}
\begin{equation}
(\bar\alpha_{j+1/2}^n +\bar\alpha_{j-1/2}^n)u_j^n = \frac{1}{2}u_j^n\left(4(\bar\al_j^n)+\Delta{x}^2\frac { \partial^2{\bar\al_j^n} } {\partial{x^2}} + \frac {\D{x^4}}{12}\frac{ \pl^4{\bar\al_j^n} } { \pl{x^4 } }+ O(\Delta{x^6})\right)
\end{equation}
Also
\begin{equation*}
\begin{split}
&\bar\alpha_{j+1/2}^n u_{j+1}^n =\\
&\begin{aligned}
&\frac{1}{2}\left(2\bar\al_j^n + \D{x}\frac{\pl{\bar\al_j^n} } {\pl{x} } +\frac {\D{x^2}}{2}\frac{ \pl^2{\bar\al_j^n} } { \pl{x^2 }} + \frac {\D{x^3}}{6}\frac{ \pl^3{\bar\al_j^n} } { \pl{x^3 } } + \frac {\D{x^4}}{24}\frac{ \pl^4{\bar\al_j^n} } { \pl{x^4 } } + O(\D{x^5})\right)\\
&\left( u_j^n + \D{x}\frac{\pl{u_j^n} } {\pl{x} } +\frac {\D{x^2}}{2}\frac{ \pl^2{u_j^n} } { \pl{x^2 }} + \frac {\D{x^3}}{6}\frac{ \pl^3{u_j^n} } { \pl{x^3 } } + \frac {\D{x^4}}{24}\frac{ \pl^4{u_j^n} } { \pl{x^4 } } + O(\D{x^5})\right)
\end{aligned}
\end{split}
\end{equation*}
and
\begin{equation*}
\begin{split}
&\bar\alpha_{j-1/2}^n u_{j-1}^n = \\
&\begin{aligned} 
&\frac{1}{2}\left(2\bar\al_j^n - \D{x}\frac{\pl{\bar\al_j^n} } {\pl{x} } +\frac {\D{x^2}}{2}\frac{ \pl^2{\bar\al_j^n} } { \pl{x^2 }} - \frac {\D{x^3}}{6}\frac{ \pl^3{\bar\al_j^n} } { \pl{x^3 } } + \frac {\D{x^4}}{24}\frac{ \pl^4{\bar\al_j^n} } { \pl{x^4 } } 
+ O(\D{x^5})\right)\\
&\left( u_j^n - \D{x}\frac{\pl{u_j^n} } {\pl{x} } +\frac {\D{x^2}}{2}\frac{ \pl^2{u_j^n} } { \pl{x^2 }} - \frac {\D{x^3}}{6}\frac{ \pl^3{u_j^n} } { \pl{x^3 } } + \frac {\D{x^4}}{24}\frac{ \pl^4{u_j^n} } { \pl{x^4 } } + O(\D{x^5}) \right)
\end{aligned}
\end{split}
\end{equation*}
multiplying the above terms, adding and then simplifying gives
\begin{equation*}
\begin{split}
\bar\alpha_{j+1/2}^n u_{j+1}^n+ \bar\alpha_{j-1/2}^n u_{j-1}^n = 
\left(2\bar\al_j^nu{_j^n}\right) +\frac{1}{2}\D{x^2}\left(2\bar\al_j^n\frac{\pl{^2u_j^n }}{\pl{x^2}} +2\frac{\pl{\bar\al_j^n }}{\pl{x}}\frac{\pl{u_j^n }}{\pl{x}} + u_j^n\frac{\pl{^2\bar\al_j^n }}{\pl{x^2}}\right) +\\ \frac{1}{2}\D{x^4}\left( \frac{ 1 }{6 }\bar\al_j^n\frac{ \pl^4u_j^n }{ \pl{x^4 }}+ \frac{1}{3} \frac{\pl{\bar\al_j^n } } { \pl{x} }\frac{\pl{^3u_j^n }}{\pl{x^3}} + \frac{1}{2} \frac{\pl^2{\bar\al_j^n } } { \pl{x^2} }\frac{\pl{^2u_j^n }}{\pl{x^2}}+\frac{1}{3} \frac{\pl^3{\bar\al_j^n } } { \pl{x^3} }\frac{\pl{u_j^n }}{\pl{x}}+ \frac{1}{12}u_j^n\frac{\pl^4{\bar\al_j^n } } { \pl{x^4} } \right) + O(\D{x^6})
\end{split}
\end{equation*}
Combining the terms
\begin{equation*}
\begin{split}
&\bar\alpha_{j+1/2}^n u_{j+1}^n+ \bar\alpha_{j-1/2}^n u_{j-1}^n - (\bar\alpha_{j+1/2}^n +\bar\alpha_{j-1/2}^n)u_j^n =\\
&\begin{aligned}
&\left(2\bar\al_j^nu{_j^n}\right) + \frac{1}{2}\D{x^2}\left(2\bar\al_j^n\frac{\pl{^2u_j^n }}{\pl{x^2}} +2\frac{\pl{\bar\al_j^n }}{\pl{x}}\frac{\pl{u_j^n }}{\pl{x}} + u_j^n\frac{\pl{^2\bar\al_j^n }}{\pl{x^2}}\right) + \\& \frac{1}{2}\D{x^4}\left( \frac{ 1 }{6 }\bar\al_j^n\frac{ \pl^4u_j^n }{ \pl{x^4 }}+ \frac{1}{3} \frac{\pl{\bar\al_j^n } } { \pl{x} }\frac{\pl{^3u_j^n }}{\pl{x^3}} + \frac{1}{2} \frac{\pl^2{\bar\al_j^n } } { \pl{x^2} }\frac{\pl{^2u_j^n }}{\pl{x^2}}+\frac{1}{3} \frac{\pl^3{\bar\al_j^n } } { \pl{x^3} }\frac{\pl{u_j^n }}{\pl{x}}+ \frac{1}{12}u_j^n\frac{\pl^4{\bar\al_j^n } } { \pl{x^4} } \right) + O(\D{x^6})- \\& \frac{1}{2}u_j^n\left(4(\bar\al_j^n)+\Delta{x}^2\frac { \partial^2{\bar\al_j^n} } {\partial{x^2}} + \frac {\D{x^4}}{12}\frac{ \pl^4{\bar\al_j^n} } { \pl{x^4 } }+ O(\Delta{x^6})\right)
\end{aligned}
\end{split}
\end{equation*}
Giving
\begin{equation*}
\begin{split}
&\bar\alpha_{j+1/2}^n u_{j+1}^n+ \bar\alpha_{j-1/2}^n u_{j-1}^n - (\bar\alpha_{j+1/2}^n +\bar\alpha_{j-1/2}^n)u_j^n =\\
&\begin{aligned}
&\D{x^2}\left(\bar\al_j^n\frac{\pl{^2u_j^n }}{\pl{x^2}} +\frac{\pl{\bar\al_j^n }}{\pl{x}}\frac{\pl{u_j^n }}{\pl{x}} \right) + \\
&\frac{1}{2}\D{x^4}\left( \frac{ 1 }{6 }\bar\al_j^n\frac{ \pl^4u_j^n }{ \pl{x^4 }}+ \frac{1}{3} \frac{\pl{\bar\al_j^n } } { \pl{x} }\frac{\pl{^3u_j^n }}{\pl{x^3}} + \frac{1}{2} \frac{\pl^2{\bar\al_j^n } } { \pl{x^2} }\frac{\pl{^2u_j^n }}{\pl{x^2}}+\frac{1}{3} \frac{\pl^3{\bar\al_j^n } } { \pl{x^3} }\frac{\pl{u_j^n }}{\pl{x}}\right) + O(\D{x^6})
\end{aligned}
\end{split}
\end{equation*}
Rewriting equation (4)
\begin{equation*}
\begin{split}
&\tau_j^n = \frac{\pl{u_j^n }}{\pl{t}} + \frac{\D{t}}{2}\frac{\pl^2{u_j^n }}{\pl{t^2}} -\left(\bar\al_j^n\frac{\pl{^2u_j^n }}{\pl{x^2}} +\frac{\pl{\bar\al_j^n }}{\pl{x}}\frac{\pl{u_j^n }}{\pl{x}} \right) + \\ 
&\begin{aligned}
\frac{1}{2}\D{x^2}\left( \frac{ 1 }{6 }\bar\al_j^n\frac{ \pl^4u_j^n }{ \pl{x^4 }}+ \frac{1}{3} \frac{\pl{\bar\al_j^n } } { \pl{x} }\frac{\pl{^3u_j^n }}{\pl{x^3}} + 
\frac{1}{2} \frac{\pl^2{\bar\al_j^n } } { \pl{x^2} }\frac{\pl{^2u_j^n }}{\pl{x^2}}+\frac{1}{3} \frac{\pl^3{\bar\al_j^n } } { \pl{x^3} }\frac{\pl{u_j^n }}{\pl{x}}\right) + 
O(\D{t^2}) + O(\D{x^4})
\end{aligned}
\end{split}
\end{equation*}
Since 
\begin{equation*}
\frac{\pl{u_j^n }}{\pl{t}} - \left(\bar\al_j^n\frac{\pl{^2u_j^n }}{\pl{x^2}} +\frac{\pl{\bar\al_j^n }}{\pl{x}}\frac{\pl{u_j^n }}{\pl{x}} \right) = \frac {\pl} {\pl{x}}(\bar\al_j^n\frac{ \pl{u_j^n} } {\pl{x } } ) =0
\end{equation*}
as $ u_j^n $ satisfies the PDE (equation $(\ref{eq:pde})$).
Therefore for the explicit scheme $(\ref{eq:expdiff})$ and writing $u_j^n$ as $u(x_j,t_n)$ from the earlier definition, the truncation error may be expressed as
\begin{equation}
\label{eq:exptru2}
\begin{split}
&\tau(x_j,t_n) = (1/2)\D{t}A + (1/2)\D{x^2}B + O(\D{t^2}) + O(\D{x^4})\\ 
&\begin{aligned}
\text{where\qquad}
A &= u_{tt}(x_j,t_n)\\
B &= \frac{ 1 }{6 }\al(u(x_j,t_n))u_{xxxx}(x_j,t_n)+ \frac{1}{3} \al_x(u(x_j,t_n ))u_{xxx}(x_j,t_n ) \\ &+\frac{1}{2}\al_{xx}(u(x_j,t_n))u_{xx}(x_j,t_n) +\frac{1}{3}\al_{xxx}(u(x_j,t_n))u_{x}(x_j,t_n)
\end{aligned}
\end{split}
\end{equation}
Following a similar process it can be concluded that for the implicit scheme $(\ref{eq:impdiff})$ the truncation error is
\begin{equation}
\label{eq:imptru}
    \begin{split}
        &\tau(x_j,t_{n+1}) = (1/2)\D{t}C + (1/2)\D{x^2}D  +  O(\D{t^2}) + O(\D{x^4})\\ 
        &\begin{aligned}
            \text{where\qquad}
            C &=  u_{tt}(x_j,t_{n+1})\\
            D &= \frac{ 1 }{6 }\al(u(x_j,t_n))u_{xxxx}(x_j,t_{n+1})+ \frac{1}{3} \al_x(u(x_j,t_{n} ))u_{xxx}(x_j,t_{n+1} ) \\ &+\frac{1}{2}\al_{xx}(u(x_j,t_{n})u_{xx}(x_j,t_{n+1})) +\frac{1}{3}\al_{xxx}(u(x_j,t_n))u_{x}(x_j,t_{n+1})
        \end{aligned}
    \end{split}
\end{equation}
\end{appendices}


%-------------end of document body----------
\end{document}

