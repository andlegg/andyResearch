\documentclass[11pt]{article}
%-------------preamble begins-------------------
%\usepackage[utf8]{inputenc}
\usepackage{cite}
\usepackage[textwidth=6.2in,textheight=9.5in]{geometry}
\usepackage{amsfonts}
\usepackage{lmodern}
%\usepackage{amsmath}
\usepackage{ amssymb }
%\usepackage{fancyhdr} 
\usepackage{array,booktabs}
\usepackage{csvsimple}
\usepackage{pdfpages}
%\usepackage[table,xcdraw]%{xcolor}
\usepackage{mathtools, nccmath}
%\usepackage{refcheck}
\usepackage{subfigure}
\usepackage{titlesec}
\usepackage{appendix}
\usepackage{enumerate}
%\setcounter{secnumdepth}{4}
\titleformat{\paragraph}
{\normalfont\normalsize\bfseries}{\theparagraph}{1em}{}
\titlespacing*{\paragraph}
{0pt}{3.25ex plus 1ex minus .2ex}{1.5ex plus .2ex}
\newcommand{\pl}{\partial}
\newcommand{\D}{\Delta}
\newcommand{\ul}{\underline}
\newcommand{\bigv}{\mathbb{V}}
\newcommand{\Om}{\Omega}
%\fancyfoot{\thepage}
\renewcommand{\labelitemi}{$\bullet$}
\renewcommand*\familydefault{\sfdefault}
\sffamily
\newcommand{\rd}{\mathrm{d}}
\newcommand{\al}{\mathrm{\alpha}}
\title{A priori error estimation modelling non-linear diffusion using finite elements and local mass conservation v5}
\author{Andy Legg}
\date{Nov 2017}
%
%-------------preamble ends---------------------
\begin{document}
%-------------start of document body----------
\maketitle
\section{Summary}
This report proposes an a priori error estimate for the solution of a non-linear diffusion equation. The solution uses a finite element algorithm with local mass conservation, following the  approach of \cite{Baines2011}. The moving domain $\Om(t)\subset\mathbb{R}^2$ is divided into a finite number of triangular elements $T_h$.  The velocity $\ul{v}$ is approximated from a discrete form of the governing PDE with standard piecewise linear basis functions defined on each element $K\in T_h$. At a given time known values of saturation $s$ are used to calculate $\ul{v}$. The velocity $\ul{v}$ is used to approximate the revised mesh point positions $\ul{x}$ after a time period $\D{t}$. Local mass conservation is applied over each element $K$ to approximate the updated values of $s$ across  $T_h$. The process is repeated for an appropriate number of time steps. The local error for on $K \in T_h$ is estimated at a given time step $t=n\D{t}$, in the following way:
\begin{itemize}
\item calculate the error in $\ul{v}$;
\item determine the error in the movement of any mesh point $\ul{x};$
\item determine the error in calculation of $s$.
\end{itemize}
The local errors from each of the above stages are summed over every $K \in T_h$ to determine the global error, which was found to be second order in space. A reference problem with a known analytic solution was used to compare the computational order of accuracy with the a priori error estimate. 
%%
\section{Introduction}
The following PDE is used to model the saturation $s(\underline{x},t)$, of a contaminant applied to a porous medium: 
\begin{equation}
\label{eq:pde}
\frac{\partial{s}}{\partial{t}} = \nabla\cdot\left(\al(s)\nabla{s}\right) 
\end{equation}
with $s =s(\ul{x},t)$ and $\al(s) >0, \text{ for } \underline{x} \in \Omega{(t)}\subset\mathbb{R}^2, \;t\in(0,T]$. The initial condition is $s(\ul{x},0) = g(\ul{x})$ and the boundary condition $(s,\al{(s)})|_{\pl\Om}=0$.  Let $s \in H^{2}(\Omega)$ and assume the flow within the porous medium is irrotational, so that $\nabla\times\underline{v} = \underline{0}$ where $\underline{v}(\ul{x},t)$ is the internal flow velocity that preserves the local mass $\int_{\Om}sd\Om.$ \\
\section{Mass Conservation}
Consider the rate of change of total mass at any time $t$. Using Reynolds Transport Theorem, 
$$\frac{d}{dt}\int_{\Om(t)}sd\Om = \int_{\Om(t)}\frac{\pl{s}}{\pl{t}}d\Om + \oint_{\pl{\Om(t)}}s\ul{v}\cdot\hat{\ul{n}}dS,  $$
where $\ul{v}$ is the boundary velocity and $\hat{\ul{n}}$ the unit normal vector on the boundary $\pl{\Om}$. 
From $(\ref{eq:pde})$,
$$\frac{d}{dt}\int_{\Om(t)}sd\Om= \int_{\Om(t)}\nabla\cdot\left(\al(s)\nabla{s}\right) + \oint_{\pl{\Om(t)}}s\ul{v}\cdot\hat{\ul{n}}dS. $$
Applying the Divergence Theorem, this may be written,
$$\frac{d}{dt}\int_{\Om(t)}sd\Om = \oint_{\pl{\Om(t)}}\left(  \al(s)\nabla{s} +  s\ul{v}\right)\cdot\hat{\ul{n}}dS. $$  
Since $\al(s), s =0$ on the boundary, we have zero flux and so the RHS of the above equation is zero. Hence
$$  \frac{d}{dt}\int_{\Om(t)}sd\Om =0, \forall \;t>0 .$$This implies that the total mass remains constant for all time. It is a property of equation $(\ref{eq:pde})$ and the given boundary conditions, that mass is conserved over time.
%%
\section{Velocity Equation}
An equation for the flow velocity may be derived using equation ($\ref{eq:pde}$) and the mass conservation equation:
\begin{equation}
\label{eq:masscon}
\frac{\pl{s} }{\pl{t} } + \nabla.(s\underline{v}) =0.
\end{equation}
Subtracting equation $(\ref{eq:masscon})$ from $(\ref{eq:pde})$ gives:
\begin{equation}
\label{eq:vel0}
\nabla\cdot\left(\al(s)\nabla{s}\right) = - \nabla\cdot(s\underline{v}),
\end{equation}
from which if $\ul{v}$ is irrotational it can be shown, refer to Appendix $\ref{appendix:helmholtz}$, that
\begin{equation}
\label{eq:vel}
\underline{v} =  -\left(\al(s)/s\right)\nabla{s}.
\end{equation}
Finding $\ul{v}$ directly from ($\ref{eq:vel}$) using piecewise linear approximations to $s$ over the elements would give discontinuities in $\ul{v}$ at the mesh points. One solution to this problem is to solve a weak form of equation $(\ref{eq:vel})$ using finite elements.
\section{A Weak Form of the Velocity Equation}
\subsection{Existence and uniqueness of solution for continuous case}
In 2-D Cartesian coordinates let $\ul{v} = (v_1(x,y),v_2(x,y))$. Introduce the function space $ \bigv = H^1(\Om)\equiv\{\ul{v}: \ul{v}, \ul{v}_x, \ul{v}_y  \in [{L}^2(\Omega)]^2\}$ with inner product and norm defined below:
$$(\ul{v},\ul{w})_{H^{1}(\Om)} = \int_{\Om}(\ul{v}\cdot\ul{w}+ \ul{v}_x\cdot\ul{w}_x + \ul{v}_y\cdot\ul{w}_y )d\Om\quad \forall \;\ul{v},\; \ul{w} \in\bigv$$
$$ \|\ul{v}\|_{H^1(\Om)} =\left[ \int_{\Om} (|\ul{v}|^2 + |\ul{v}_x|^2 +|\ul{v}_y|^2 )d\Om\right]^{1/2}\; \forall \;\ul{v}\in\bigv$$
To solve this weak form of equation $\ref{eq:vel}$ we need to find $\ul{v} \in \bigv \text{ s.t. }$
\begin{equation}
\label{eq:weakvel}
\int_{\Omega(t)} \ul{v}\cdot\ul{w}d{\Omega} =  -\int_{\Omega(t)}(\al(s)/s)\nabla{s}\cdot\ul{w}d{\Omega}
\end{equation}
$\forall\; \ul{w} \in \bigv$.
Define the bilinear form $$a(\ul{v},\ul{w}) = \int_{\Om}\ul{v}\cdot\ul{w}d{\Om}$$ and linear form $$L(\ul{w}) =  \int_{\Om}\ul{f}\cdot\ul{w}d\Om, \text{ with } \ul{f} = -(\al(s)/s)\nabla{s}.$$
Hence equation ($\ref{eq:weakvel}$) may be written:
\begin{equation}
\label{eq:weakvel3a}
a(\ul{v},\ul{w}) = L(\ul{w}),
\end{equation}
$\forall\; \ul{w} \in \bigv.$
To show that equation ($\ref{eq:weakvel3a}$) has a unique solution, the Lax Milgram theorem \cite{Lax2016}, requires the following conditions to be satisfied:
\begin{enumerate}[(i)]
\item $a(\cdot,\cdot)$ is symmetric;
\item $a(\cdot,\cdot)$ is continuous, i.e. $\exists\; \gamma >0\; \text{ s.t. } |a(\ul{v},\ul{w})| \leq \gamma\|\ul{v}\|_{\bigv}\|\ul{w}\|_{\bigv} \quad \forall \ul{v},\;\ul{w} \in \bigv$;
\item $a(\cdot,\cdot)$ is V-elliptic, i.e. $\exists\; \zeta > 0 \; \text{ s.t. } \; a(\ul{v},\ul{v}) \geq \zeta\|  \ul{v}   \|_{\bigv}^2\quad \forall \; \ul{v}\in\bigv$; and 
\item $L$ is continuous, i.e. $ \exists\; \Lambda > 0 \; \text{ s.t. } \; |L(\ul{v})| \leq \Lambda\|\ul{v}   \|_{\bigv}\quad \forall\;\ul{v} \in \bigv.$
\end{enumerate}
Since $a(\ul{v},\ul{w}) = \int_{\Om}\ul{v}\cdot\ul{w}d{\Om}= \int_{\Om}\ul{w}\cdot\ul{v}d{\Om}= a(\ul{w},\ul{v})$ it follows that $a(\cdot,\cdot)$ is symmetric.\\
The Cauchy-Schwarz inequality gives 
$|(\ul{v},\ul{w})_{\bigv}| \leq \|  \ul{v}   \|_{\bigv}\|  \ul{w}   \|_{\bigv}    .$
Therefore with reference to condition (ii) and choosing $\gamma = 1, a(\cdot,\cdot)$ is continuous.\\   
Since $a(\ul{v},\ul{v})_{\bigv}= \int_{\Om}|\ul{v}|^2d\Om = (\ul{v},\ul{v})_{\bigv} =     \|  \ul{v}   \|_{\bigv}^2      $ and taking $\zeta = 1$ condition (iii) is satisfied and so $a(\cdot,\cdot)$ is V-elliptic.\\
Consider $|L(\ul{v})| =   \int_{\Om}\ul{f}\cdot\ul{v}d\Om$. By application of the Cauchy-Schwartz inequality, $|L(\ul{v})| = (\ul{f},\ul{v}) \leq \|\ul{f}\|_{\bigv}\|\ul{v}\|_{\bigv}.$  Recalling that $\ul{f} = -(\al(s)/s)\nabla{s}$, if in condition (iv) we choose $\Lambda = \|\ul{f}\|_{\bigv}$, $L$ is continuous providing $f\ne0$.\\
Hence there is a unique solution to equation $\ref{eq:weakvel3a}$.
\subsection{Existence and uniqueness of solution for discrete case}
Let $\bigv_h$ be an $M$ (finite) dimensional subspace of $\bigv$. Let $\{\ul{\phi_1},\dots,\ul{\phi_M}\}$ be a basis for $\bigv_h$. Therefore $\ul{\phi_i}\in \bigv_h \text{ for } i = 1,\dots,M$ and any $\ul{w}\in \bigv_h$ has the representation 
\begin{equation}
\label{eq:lineq}
\ul{w} = \sum_{j=1}^M\eta_j\ul{ \phi_j} \text{ where }\eta_j \in \mathbb{R}.
\end{equation}
In a discrete representation of equation (\ref{eq:weakvel3a}) we want to find $$\ul{v_h}\in \bigv_h \text{ s.t } a(\ul{v_h},\ul{w}) = L(\ul{w}) \quad \forall \; \ul{w} \in \bigv_h.$$
Since $\ul{\phi_i} \in \bigv_h$
\begin{equation}
\label{eq:discrete1}
a(\ul{v_h},\ul{\phi_i}) = L(\phi_i) \quad \text{ for } i = 1,\dots,M.
\end{equation}
Let $$\ul{v_h} = \sum_{j=1}^M\xi_j\ul{ \phi_j}\;\text{ where } \ul{\xi} =  \{\ul{\xi_1},\dots,\ul{\xi_M}\} \text{ and } \ul{\xi_j} \in \mathbb{R}$$ then equation (\ref{eq:discrete1}) becomes
$$a( \sum_{j=1}^M\xi_j\ul{ \phi_j},\ul{\phi_i}) = L(\ul{\phi_i})\quad \text{for } i=1,\dots,M.$$
Giving
$$\sum_{j=1}^Ma( \ul{ \phi_j},\ul{\phi_i})\xi_j = L(\ul{\phi_i})\quad \text{for } i=1,\dots,M.$$
This equation may be written in matrix form,
\begin{equation}
\label{eq:discrete22}
A\ul{\xi} = \ul{b},
\end{equation}
where $$\ul{b} = (b_i)\in \mathbb{R}^M\; \text{ with } b_i = L(\ul{\phi_i} \text{ and } A = (a_{i,j}) \text{ is an } M\times{M} \text{ matrix with elements } a_{i,j} = a(\ul{\phi_i},\ul{\phi_j}).$$
Note that from the definition of $A$ we can conclude that it is symmetric. From equation $(\ref{eq:lineq})$,
$$a(\ul{w},\ul{w}) = a(\sum_{i=1}^M\eta_i\ul{ \phi_i},\sum_{j=1}^M\eta_j\ul{ \phi_j}) = \sum_{i=1}^M \sum_{j=1}^M \eta_ia(\ul{\phi_i },\ul{\phi_j })\eta_j = \ul{\eta}\cdot{A}\ul{\eta},$$
$$L(\ul{w}) = L(\sum_{i=1}^M\eta_i\ul{ \phi_i}) = \sum_{i=1}^M\eta_iL(\ul{\phi_i })=\ul{b}\cdot\ul{\eta}.$$
Since from the Lax Milgram Theorem condition (iii) we have $a(\ul{w},\ul{w}) \geq \zeta\|\ul{w}\|_{\bigv}^2 > 0, \text{ providing } \ul{w} \neq 0,$ then $\ul{\eta}\cdot{A}\ul{\eta} > 0.$
This implies that the matrix $A$ is positive definite and so non-singular. Hence there is a unique solution to the system of linear equations $(\ref{eq:discrete22})$ and so to the discrete form of the problem.
\subsection{Error Estimate in velocity using piecewise linear approximation}
Consider the case where $\bigv = H^1(\Om)$ and \\$V_h = \{\ul{v}\in \bigv\;: \ul{v}\;\text{ is a piecewise linear function defined each triangle}\; K \in T_h\} \text{ where } \\T_h \text{ is a triangulation of } \Om\subset\mathbb{R}^2.$
For $K\in{T_h}$ define:
$$h_K = \text{ longest side of } K,$$
$$\rho_K = \text{ the diameter of the circle inscribed in } K,$$
$$h = \max_{K\in{T_h}}h_K.$$
A condition 
\begin{equation}
\label{eq:beta1}
\rho_K/h_K\geq\beta>0,\quad \forall\;K\in{T_h},
\end{equation}
is required to ensure that the triangles $K\in{T_h}$ are not allowed to be arbitrarily thin or have angles that are too small.
Under these conditions, \cite{Johnson2012} and \cite{Dupont1980}, it can be proved that the error bounds on $\ul{v}\in\bigv,$ approximated by a piecewise linear function $\ul{v}_h\in\bigv_h$ on $K\in{T_h}$ are given by:
\begin{equation}
\label{eq:error1}
\|\ul{v}-\ul{v_h}\|_{L^2(K)} \leq Ch_K^2|\ul{v}|_{H^2(K)},
\end{equation}
\begin{equation}
\label{eq:error2}
|\ul{v}-\ul{v_h}|_{H_1(K)} \leq (Ch_K^2/\rho_K)|\ul{v}|_{H^2(K)},
\end{equation}
where $C \in \mathbb{R}$ is a constant that is independent of $h$.
These results are used to determine the global interpolation errors, $$\|\ul{v}-\ul{v_h}\|_{L^2(\Om)} \text{ and } |\ul{v}-\ul{v_h}|_{H_1(\Om)}.$$
Summing error over $K\in{T_h}$ and using inequality (\ref{eq:error1})
Therefore the global error bound is
\begin{equation}
\label{eq:error3}
\|\ul{v}-\ul{v_h}\|_{L^2(\Om)} \leq Ch^2|\ul{v}|_{H^2(\Om)}.
\end{equation}
Similarly for the first derivatives of the global error, summing over $K\in{T_h}$ and using inequalities $(\ref{eq:error2})$ and $(\ref{eq:beta1})$,
$$|\ul{v}-\ul{v_h}|_{H_1(\Om)}^2 \leq \sum_{K\in{T_h}}C^2(h_K^4/\rho_K^2)|\ul{v}|_{H^2(K)}^2 \leq \sum_{K\in{T_h}}C^2(h_K^2/\beta^2)|\ul{v}|_{H^2(K)}^2 \leq C^2(h^2/\beta^2)|\ul{v}|_{H^2(\Om)}^2.$$
Therefore the global error bound on the first derivatives is
\begin{equation}
\label{eq:error4}
|\ul{v}-\ul{v_h}|_{H_1(\Om)} \leq C(h/\beta)|\ul{v}|_{H^2(\Om)},
\end{equation}
In summary both the continous and discrete versions of the weak velocity equation have unique solutions. The a priori error bounds show that the convergence rate of the method is second order in space if we use piecewise linear interpolation functions to determine the approximate solution. \\
In the context of the solution to equation ($\ref{eq:pde}$) the mesh points $\ul{x}\in \Om(t)$ move with time. In particular $h_K$ and $h$ will vary at each time step. Let time $t = n\D{t}$, so $K^n \in T_h^n$ represents the element $K$ in triangulation $T_h$ after time $t$. We rewrite ($\ref{eq:error1}$) and ($\ref{eq:error2}$) after time $t$ as:
\begin{equation}
\label{eq:error3_1}
\|\ul{v}^n-\ul{v_h}^n\|_{L^2(K^n)} \leq C{h_K^n}^2|\ul{v}^n|_{H^2(K^n)},
\end{equation}
\begin{equation}
\label{eq:error4_1}
|\ul{v}^n-\ul{v_h}^n|_{H_1(K^n)} \leq D{h_K^n}^2|\ul{v}^n|_{H^2(K^n)},
\end{equation}
where $C, D \in \mathbb{R}$ are constants and $\rho_K^n$ is incorporated into $D$.
\section{Error estimate in movement of mesh points}
Consider $\ul{x}(t) \in K$, where $\ul{x} = (r(t),z(t))$, using a Taylor Series expansion, after time $\D{t}$:
\begin{equation}
\label{eq:pos1}
\ul{x}(t+\D{t}) = \ul{x}(t) + \D{t}\dot{\ul{x}}(t) +( \D{t}^2/2)\ddot{\ul{x}}(\xi),
\end{equation}
where $\xi \in (t, t+\D{t})$.\\
For the discrete numerical model, $\ul{x}_h \in \Omega_h(t)$, after time $t + \D{t}$ and adopting a superscript notation to represent the time step $n$, we use a linear approximation to determine $\ul{x}_h^{n+1}$:
\begin{equation}
\label{eq:pos2}
\ul{x}_h^{n+1} = \ul{x}_h^n + \D(t)\dot{\ul{x}}_h(t).
\end{equation}
Writing $\ul{x}(t+\D{t})$ as $\ul{x}^{n+1}$ and subtracting ($\ref{eq:pos2}$) from ($\ref{eq:pos1}$) and defining the error in the approximation as $\ul{ex}^n = \ul{ex}^n - \ul{ex}_h^n$,
\begin{equation}
\label{eq:err1}
\ul{ex}^{n+1} = \ul{ex}^n + \D{t}(\dot{\ul{x}}^n-\dot{\ul{x}}_h^n) + ( \D{t}^2/2)\ddot{\ul{x}}(\xi^n).
\end{equation}
Using successive time steps $0,\dots,n+1$ and writing $\ul{v}^n = \dot{\ul{x}}^n$ and $\dot{\ul{v}}^n = \ddot{\ul{x}}^n$  it is possible to derive the following relationship:
\begin{equation}
\label{eq:err2}
\ul{ex}^{n} = \ul{ex}^0 + \D{t}\sum_{j=1}^{n}(\ul{v}^{j-1}-\ul{v}_h^{j-1}) + ( \D{t}^2/2)\sum_{j=1}^{n}\dot{\ul{v}}(\hat{\xi}^{j-1}),
\end{equation}
where $\hat{\xi}^n \in (t,t+\D{t}).$ This gives
$$\|\ul{ex}\|_{L^{2}(K^n)} \leq \| \ul{ex}\|_{L^{2}(K^0)} + \|\D{t}\sum_{j=1}^{n}(\ul{v}^{j-1}-\ul{v}_h^{j-1})\|_{L^{2}(K^n)} + \|( \D{t}^2/2)\sum_{j=1}^{n}\dot{\ul{v}}(\hat{\xi}^{j-1})\|_{L^{2}(K^n)}.$$
Therefore
$$\|\ul{ex}\|_{L^{2}(K^n)} \leq \| \ul{ex}\|_{L^{2}(K^0)} + n\D{t}\max_{1 \leq j \leq n}\|(\ul{v}^{j-1}-\ul{v}_h^{j-1})\|_{L^{2}(K^{j-1})} + ( n\D{t}^2/2)\max_{1\leq j \leq n}\|\dot{\ul{v}}(\hat{\xi}^{j-1})\|_{L^{2}(K^{j-1})}.$$
Using equation $(\ref{eq:error3})$ we obtain,
\begin{equation}
\label{eq:err3}
\|\ul{ex}\|_{L^{2}(K^n)} \leq \| \ul{ex}\|_{L^{2}(K^0)} + nC\D{t}\max_{1 \leq j \leq n}\left({h_K^j}^2|\ul{v}^{j-1}|_{H^2(K^{j-1})}\right)  + ( n\D{t}^2/2)\max_{1\leq j \leq n}\|\dot{\ul{v}}(\hat{\xi}^{j-1})\|_{L^{2}(K^{j-1})}.
\end{equation}
To simplify, suppose $|\ul{ex}_0| =    \|\ul{ex}\|_{L^{2}(K^0)}, \;n\D{t} \sim 1,$
$$\max_{1\leq{j}\leq {n}}h_K^j = \hat{h},\;\max_{1\leq{j}\leq{n}}|\ul{v}^{j-1}|_{H^2(K^{j-1})} = |\hat{\ul{v}}| \text{ and }  \max_{1\leq j \leq n}\|\dot{\ul{v}}(\hat{\xi}^{j-1})\|_{L^{2}(K^{j-1})} =   |\hat{\dot{\ul{v}}}|                                       $$  then
\begin{equation}
\label{eq:err4}
\|\ul{ex}\|_{L^{2}(K^n)} \leq{|\ul{ex}_0|} + C\hat{h}^2|\hat{\ul{v}}| + (\D{t}/2)|\hat{\dot{\ul{v}}}|.
\end{equation}
Simplifying further
\begin{equation}
\label{eq:err5}
\|\ul{ex}\|_{L^{2}(K^n)} \leq C_0 + C_1\hat{h}^2 + C_2\D{t},
\end{equation}
where $C_0, C_1 \text{ and } C_2$ are constants and it is assumed that $\ul{v}$ and $\dot{\ul{v}}$ are bounded on $\Om(t)$. If there is no initial error in $\ul{x}$ then $C_0=0$.
%%
\section{Error estimate in mass recovery}
Define function space $ \bigv = H^1(\Om)\equiv\{s: s, s_x, s_y \in [{L}^2(\Omega)]\}$ with inner product and norm defined below:
$$(s,w)_{H^{1}(\Om)} = \int_{\Om}(sw+ s_xw_x + s_yw_y )d\Om\quad \forall \; s, w \in\bigv; \text{ and }$$
$$ \|s\|_{H^1(\Om)} =\left[ \int_{\Om} (|s|^2 + |s_x|^2 +|s_y|^2 )d\Om\right]^{1/2}\; \forall \; s\in\bigv.$$
Following the approach of reference \cite{Baines2011}, using the weak form of conservation of mass at time $t>0$:
\begin{equation}
\label{eq:mass1}
\int_{\Om(t)}spd{\Om} = M^0, \; \forall \;p\in\bigv, 
\end{equation}
where $M^0$ is the mass at time $t=0$.
Suppose $s_h \in\bigv_h \subset \bigv$, is the approximate solution, then
\begin{equation}
\label{eq:mass2}
\int_{\Om_h(t)}s_hwd{\Om} = M_h^0, \; \forall \; w\in\bigv_h, 
\end{equation}
where $M_h^0$ is the mass at time $t=0$.
If $s_h = 0 \; \forall \; \ul{x} \notin \Om_h,$ and $\Om_h(t) \subset \Om(t),$ since $w \in \bigv,$
$$\int_{\Om_h(t)}ws_h d\Om = \int_{\Om(t) }ws_hd\Om.$$
Therefore changing the domain in ($\ref{eq:mass2}$) to $\Om(t) $ and subtracting ($\ref{eq:mass2}$) from ($\ref{eq:mass1}$),
\begin{equation}
\label{eq:mass3}
\int_{\Om(t)}(s-s_h)wd{\Om} = eM^0, \; \text{ for } w\in\bigv_h, 
\end{equation}
where $eM^0 = M^0 - M_h^0,$ the initial error in the mass.
Consider
$$\|s-s_h\|_{L^2(\Om(t))}^2 = \int_{\Om(t)}(s-s_h)(s-s_h)d\Om,$$
using ($\ref{eq:mass3}$),
$$\|s-s_h\|_{L^2(\Om(t))}^2 = \int_{\Om(t)}(s-s_h)(s-s_h)d\Om + \int_{\Om(t)}(s-s_h)wd{\Om} - eM^0,         $$
so
$$\|s-s_h\|_{L^2(\Om(t))}^2 \leq \int_{\Om(t)}(s-s_h)(s-s_h)d\Om + \int_{\Om(t)}(s-s_h)wd{\Om} + |eM^0|.$$
Let $w = s_h - q$, where $q\in\bigv_h,$ giving,
$$\|s-s_h\|_{L^2(\Om(t))}^2 \leq \int_{\Om(t)}(s-s_h)(s-s_h+w)d\Om + |eM^0| \leq \int_{\Om(t)}(s-s_h)(s-q)d\Om +|eM^0| .$$
Applying the Cauchy-Schwartz inequality to the RHS,
$$\|s-s_h\|_{L^2(\Om(t))}^2 \leq \int_{\Om(t)}(s-s_h)d\Om \int_{\Om(t)}(s-q)d\Om + |eM^0|.$$
Applying the H{\"o}lder inequality to the RHS,
$$\|s-s_h\|_{L^2(\Om(t))}^2 \leq (1/2)\left(\int_{\Om(t)}(s-s_h)d\Om\right)^2 + (1/2)\left(\int_{\Om(t)}(s-q)d\Om\right)^2 +|eM^0|.$$
This gives
$$\|s-s_h\|_{L^2(\Om(t))}^2 \leq (1/2)\|(s-s_h)\|_{L^2(\Om(t))}^2 + (1/2)\|(s-q)\|_{L^2(\Om(t))}^2  +|eM^0| \;\forall\;q\in\bigv_h,$$
which may be simplified,
\begin{equation}
\label{eq:err_s}
\|s-s_h\|_{L^2(\Om(t))}^2 \leq \|(s-q)\|_{L^2(\Om(t))}^2  + 2|eM^0|, \;\forall\;q\in\bigv_h.
\end{equation}
If $s\in\bigv_h$ and $q \in\bigv_h$ is chosen to be a linear interpolant of $s$ on $K\in T_h$, then using the condition from inequality ($\ref{eq:error1}$), the error bound for $s$ on any $K\in{T_h}$ is given by:
\begin{equation}
\label{eq:error5}
\|s-s_h\|_{L^2(K)} \leq F\hat{h}_K^2|s|_{H^2(K)} + 2|eM_K^0|,
\end{equation}
where $F\in\mathbb{R},\hat{h}_K$ is the largest side of $K$, and condition ($\ref{eq:beta1}$) holds $\forall\;{K}\in{T}_h$.
Equation ($\ref{eq:error5}$) may be written,
\begin{equation}
\label{eq:error6}
\|s^n-s_h^n\|_{L^2(K^n)} \leq F(\hat{h}^n_K)^2|s^n|_{H^2(K^n)} + 2|eM_K^0|,
\end{equation}
to specify the error at timestep $n$.

\subsection{Global error analysis}
To determine the global error at time $t=n\D{t}$, it is necessary superimpose the following:
\begin{itemize}
\item the error in $\ul{v}^n$ which depends upon $(h_K^n)^2$, refer to inequality ($\ref{eq:error3}$); 
\item the error in the position of the mesh points $\ul{x}^n\in K^n$ which depends upon $\ul{v}^n$ and is given in inequality ($\ref{eq:err3}$); and
\item the error in the calculation of $s^n$ which depends upon $\ul{x}^n$, and is given in inequality ($\ref{eq:error6})$. 
\end{itemize}
Define $h_K^n$ and $\hat{h}_K^n$ as the longest sides of K, at the beginning and end of time step $n$ respectively. The relationship between 
 $h_K^n$ and $\hat{h}_K^n$ is given by:
 \begin{equation}
 \label{eq:hrel}
 \hat{h}_K^n \leq h_K^n + 2\|\ul{e}_K^n\|_{L^{2}(K^n)},
 \end{equation}
 with $$\|\ul{ex}^{n}\|_{L^{2}(K^n)} \leq \| \ul{ex}^0\|_{L^{2}(K^0)} + nC\D{t}\max_{1 \leq j \leq n}{h_K^j}^2|\ul{v}^{j-1}|_{H^2(K^{j-1})}  + ( n\D{t}^2/2)\max_{1\leq j \leq n}\|\dot{\ul{v}}(\hat{\xi}^{j-1})\|_{L^{2}(K^{j-1})}, \text{ from }  (\ref{eq:err3}).$$
 Using ($\ref{eq:err3}$) in ($\ref{eq:error6})$:
\begin{equation}
\label{eq:error7}
\|s^n-s_h^n\|_{L^2(K^n)} \leq F( h_K^n + 2\|\ul{ex}^n\|_{L^{2}(K^n)}  )^2|s|_{H^2(K^n)} + 2|eM_K^0|.
\end{equation}
If $P = \text{ number of elements in } T_h, \text{ and }$ $$\bar{h} = \max_{ K^n\in{T_h}, 1\leq{j}\leq{n} }h_K^j, \text{ then }$$
\begin{align*} 
\|s^n-s_h^n\|_{L^2(\Om_h^n)} & \leq F\sum_{K\in\ T_h}( \bar{h} + 2\|\ul{ex}^n\|_{L^{2}(K^n)}  )^2\sum_{K\in\ T_h}|s|_{H^2(K^n)} + 2\sum_{K\in\ T_h}|eM_K^0|\\ 
& \leq F\sum_{K\in\ T_h}( \bar{h} + 2\|\ul{ex}^n\|_{L^{2}(K^n)}  )\sum_{K\in\ T_h}( h^n + 2\|\ul{ex}^n\|_{L^{2}(K^n)}  )\sum_{K\in\ T_h}|s|_{H^2(K^n)} + 2\sum_{K\in\ T_h}|eM_K^0| \\ 
& \leq F\left(P\bar{h}+2\|\ul{ex}^n\|_{L^2(\Om^n_h)}\right)^2|s|_{H^2(\Om^n_h)} + 2|eM^0|.\\ 
\end{align*}
This can be simplified, if $n\D{t} \sim 1, \text{ assuming initial errors } \|\ul{ex}^0\|_{L^2(\Om^n_h)}\sim 0 \text{ and } |eM^0| \sim 0, $ 
where $$|\hat{\ul{v}}| = \max_{1\leq{j}\leq{n}}|\ul{v}|_ {H^2(\Om_h^{j-1})}), \; |\hat{\dot{\ul{v}}}|=\max_{1\leq j \leq n}\|\dot{\ul{v}}(\hat{\xi})\|_{L^{2}(\Om_h^{j-1})} \text { and } |\hat{s}|=\max_{1\leq{j}\leq n}|s|_{H^2(\Om^{j-1}_h)},$$
Then
$$\|s^n-s_h^n\|_{L^2(\Om_h^n)} \leq  F\left(P\bar{h}+2 \bar{h}^2|\hat{\ul{v}}|  + ( \D{t}/2)\ |\hat{\dot{\ul{v}}}|      \right)^2|\hat{s}|.$$
Therefore the FEM/mass conservation method using piecewise linear approximation is expected to be second order in space, if $\D{t} \sim \bar{h}^2$, since the expression for the error in $s$ is of the form:
\begin{equation}
\label{eq:errest}
 \|s^n-s_h^n\|_{L^2(\Om_h^n)} \leq C_1\bar{h}^2 + C_2\D{t}(\bar{h})+ C_3\D{t}^2 + O({\bar{h}^3}),
 \end{equation}
where $C_1,C_2 \text{ and } C_3$ are constants. This is subject to the assumption that the second derivatives of $s$ and $\ul{v}$ are bounded on $\Om(t)$.
%%
\section{Comparison with computational results}
%\subsection{1D example}
%Consider a 1D problem, using $\al(s) = s$ . For this example ($\ref{eq:pde}$) reduces to
%\begin{equation}
%\label{eq:pde1}
%\frac{\partial{s}}{\partial{t}} = \frac{\pl}{\pl{x}}\left(s\frac{\pl{s}}{\pl{x}}\right) 
%\end{equation}
%with $s =s(x,t)$ and $s >0, \text{ for } x \in (0,b(t)), \;t\in(0,T]$. The initial condition is $s(x,1) = g(x)$ and the boundary condition $s(b(t),t)=0$. 
%The exact solution for this type of problem is well known and is given in \cite{Baines2006}. For this example, take
%$$g(x) = 6^{-1/3}(1 - {6}^{-2/3}x^{2}), \text{ with } s(x,t) =(6t)^{-1/3}\left(1 - (6t)^{-2/3}x^{2}\right).$$
%The approximate solution $s_h$ after a computational period $t=0.1$ was compared with the exact solution $s$. The error norm $\|s-s_h\|_{L^{2}(0,b(t))}$ and order of convergence $p$ are shown in the following table. The results support the a priori estimate that the method of solution is second order in space.
%
%\begin{table}[]
%\centering
%\caption{Error Analysis 1D problem}
%\label{Table 1}
%\begin{tabular}{|l|l|l|l|l|l|}
%\hline
%					
%J	&$\bar{h}$	&$\D{t}$	&L2 error norm	&ratio error	&p \\ \hline
%20	&0.089529139	&0.003743682	&0.000778742	&   & \\ \hline	
%40	&0.045865834	&0.000982132	&0.000222266	&0.285416967	&1.808856995 \\ \hline
%80	&0.023208689	&0.000251633	&3.82349E-05	&0.172022874	&2.539327677 \\ \hline
%\end{tabular}
%\end{table}
%


\subsection{3D example }
For reference purposes consider a 3D problem with radial symmetry, using $\al(s) = s$. Using the standard transformation from Cartesian $(x,y,z)$ to spherical polar coordinates $(r,\theta, \phi)$ and with $\frac{\pl}{\pl{\theta}},\;\frac{\pl}{\pl{\phi}}\equiv 0$, equation ($\ref{eq:pde}$) reduces to
\begin{equation}
\label{eq:pde1}
\frac{\partial{s}}{\partial{t}} = (1/r^2)\frac{\pl}{\pl{r}}\left(r^2s\frac{\pl{s}}{\pl{r}}\right) 
\end{equation}
where r is the usual radial coordinate, $s =s(r,t)$ and $s >0, \text{ for } r \in (0,b(t)), \;t\in(0,T]$. The initial condition is $s(r,1) = g(r)$ and the boundary condition $s(b(t),t)=0$. 
The exact solution for this type of problem is well known and is given in \cite{Baines2006}. For this example $$ s(r,t) =(t)^{-3/5}\left(1 - (1/10)(t)^{-2/5}r^{2}\right).$$
Following the FEM/mass conservation method described in this report the approximate solution $s_h$ after a computational period $t=0.1$ with $\D{t} = 0.1(b(t^0)/(J+1))^2$ was compared with the exact solution $s$. The error norm $\|s-s_h\|_{L^{2}(0,b(t))}$ and order of convergence $p$ are shown in Table $\ref{Table 1}$. In the table $J$ is the number of internal points in the interval $(0,b(t))$, the error ratio $R = \|s-s_{h/2}\|_{L^2}/\|s-s_h\|_{L^2}$, 
$$h = \max_{n=1}^N\left[\max_{j=1}^{J+1}{(r_{j+1}^n - r_j^n)}\right]$$ and $p = log(R)/log(1/2)$. The results show $p\sim 2$, supporting the a priori estimate given in $(\ref{eq:errest})$ that predicts the method of solution to be second order in space.

\begin{table}[]
\centering
\caption{Error Analysis 3D diffusion with radial symmetry}
\label{Table 1}
\begin{tabular}{|l|l|l|l|l|l|}
\hline
J	&$\bar{h}$	&$\D{t}$	&L2 error norm	&ratio error R	&p \\ \hline
20	&0.153811279	&0.002267574	&0.000973768	& &	\\ \hline
40	&0.078764976	&0.000594884	&0.000253287	&0.260110284	&1.942804653 \\ \hline
80	&0.039857371	&0.000152416	&6.4693E-05	&0.255413562	&1.969092961 \\ \hline
\end{tabular}
\end{table}

%
%\nocite{Lee2014}
%%\nocite{Morton1}
%\nocite{Lukyanov2012}
\bibliography{RefLib}
\bibliographystyle{plain}
\newpage
\appendix
\begin{appendices}
\appendixpage
\section{Helmholtz Fundamental Theorem of Vector Analysis}
\label{appendix:helmholtz}
Let \underline{a} be a vector field on a bounded domain $\Omega \subseteq \mathbb{R}{^n}, \text{ with }n=2,3$, which is twice continuously differentiable, and let $S$ be the surface that encloses the domain $\Omega$. Then using \cite{Griffiths} $\underline{a}$ can be decomposed into a curl-free component and a divergence-free component:
$$\underline{a} =-\nabla\phi+\nabla \times \underline {w},$$
where $\phi$ and $\underline{w}$ are respectively the scalar and vector potential of $\underline{a}$.
\subsection{Corollary 1}
If $\nabla\cdot{\underline{a}}$ and $\nabla\times{\underline{a}}$ are given the decomposition is unique.
\subsection{Corollary 2}
If  $\underline{a}$ is irrotational  $\nabla\times{\underline{a}} = 0\; \implies \exists \text{ a } \phi \text{ s.t } \underline{a} = -\nabla{\phi}$.
\subsection{Corollary 3}
If  $\underline{a}$ is solenoidal  $\nabla\cdot{\underline{a}} = 0\; \implies \exists \; \underline{w} \text{ s.t } \underline{a} = -\nabla\times\underline{w}$.
\subsection{Application}
To apply this theorem to our specific problem, we want to solve equation ($\ref{eq:vel0}$) 
$$\nabla\cdot\left(\al(s)\nabla{s} + s\underline{v}\right)=0,$$ 
for a unique $\ul{v}$. However it can be seen that if $\ul{v} = (\al{(s)}/s)\nabla{s}$ is a solution then so is $\ul{v} = (\al{(s)}/s)\nabla{s} + (1/s)\nabla\times{\ul{w}}$, for any $\ul{w}$.
If $\ul{v}$ is irrotational, as stated in our problem, using Helmholtz Corollary 2 above $ \exists \text{ a } \phi \text{ s.t } \underline{v} = \nabla{\phi}.$
This means equation ($\ref{eq:vel0}$) can be written 
\begin{equation}
\label{eq:Poisson1}
\nabla\cdot(s\underline{\nabla}{\phi}) = - \nabla\cdot(\al(s)\nabla{s}).
\end{equation}
This is a Poisson equation for $\phi$, which by the Uniqueness Theorem, refer to Appendix $\ref{appendix:poisson}$, states that if $s>0$ and a potential $\phi$ satisfies the equation and boundary
 conditions, then it is a unique solution.
Hence if $\phi$ is unique then so is $\ul{v}$, which means the solution to equation ($\ref{eq:vel0}$) is $\ul{v} = (\al{(s)}/s)\nabla{s}$.

\section{Uniqueness Theorem Poisson Equation}
\label{appendix:poisson}
Suppose we want to solve the following equation for $\phi$:
\begin{equation}
\label{eq:poisson2}
\nabla\cdot(s\nabla{\phi}) = -f(s),
\end{equation}
where $s$ is specified on a domain $\Omega$ with boundary $\delta{\Omega}$. This equation is subject to the following boundary conditions (BC):
either (i) $\phi$ is specified on $\delta{\Omega}$, or (ii) $\ul{v} = -\nabla{\phi}$ is specified on $\delta{\Omega}$.
There are no restrictions on the boundary which could be at $\infty$.\\
Suppose there are 2 solutions to ($\ref{eq:poisson2}$), $\phi_1$ and $\phi_2.$
If $\phi = \phi_1-\phi_2$, then from ($\ref{eq:poisson2}$)
then
$$\nabla\cdot(s\nabla(\phi_1)-s\nabla(\phi_2)) = 0,$$
giving $$\nabla\cdot(s\nabla(\phi_1-\phi_2)) = 0,$$
$$\nabla\cdot(s\nabla(\phi)) = 0.$$
This means we can write 
$$\phi\nabla\cdot(s\nabla{\phi}) = 0.$$
Using the identity $\nabla\cdot(c\ul{A}) = \nabla{c}\cdot\ul{A} + c\nabla\cdot{\ul{A}}$ with $c = \phi$ and $\ul{A} = s\nabla{\phi}$ gives
$$ \nabla\cdot( \phi{s}\nabla\phi ) - \nabla\phi\cdot{s}\nabla\phi = 0,$$
which implies
$$ \nabla\cdot( \phi{s}\nabla\phi ) - s(\nabla\phi)^2 = 0.$$
Integrating this equation over $\Omega$ gives
$$ \int_{\Omega}(\nabla\cdot( \phi{s}\nabla\phi ) - s(\nabla\phi)^2)d\Omega=0.$$ 
Applying the divergence theorem to the first term:
$$ \oint_{\pl{\Omega}}\phi{s}\nabla{\phi}\cdot\ul{\hat{n}}dS - \int_{\Omega}s(\nabla\phi)^2d\Omega=0.$$
Now for either BC (i) or (ii) the boundary integral is zero. Hence
$$ \int_{\Omega}s(\nabla\phi)^2d\Omega=0,$$ 
since $s>0$ and $(\nabla\phi)^2\geq 0$, this $\implies \nabla{\phi} = 0,$ and so $\phi_1 = \phi_2 + k$ with $k = 0$ for BC (i). Hence equation ($\ref{eq:poisson2}$) has a unique solution. 
\end{appendices}


%-------------end of document body----------
\end{document}





